\documentclass[11pt,oneside]{book}
\usepackage[margin=2em, bottom=3em]{geometry}
\usepackage{graphicx}
\usepackage{array}
\usepackage{ragged2e}

\pagestyle{plain}

\begin{document}

\begin{center}
    \textbf{\large THE UNIVERSITY OF SHEFFIELD}\\[1em]
    \textbf{\large School of Electrical and Electronic Engineering}\\[0.5em]
    \textbf{3\textsuperscript{rd} Year Individual Project --- Project Initialisation Document}\\[1em]

    \begin{tabular}{|>{\centering\arraybackslash}m{3cm}|>{\centering\arraybackslash}m{8cm}|}
        \hline
        Student Name & Amaan Mujawar \\ 
        \hline
        Project Title & Hardware Accelerator for RISCV Processor \\ 
        \hline
        \multicolumn{2}{|c|}{
            Supervisor \hspace{1em} \textbar \hspace{1em} Mr Neil Powell \hspace{1em} \textbar \hspace{1em} 
            Second Marker \hspace{1em} \textbar \hspace{1em} FName LName
        } \\ 
        \hline
    \end{tabular}
    \\[2em]
\end{center}

\noindent
\textbf{Project Motivation:} \\
Introduce your project and the topic area so that someone with no engineering background could understand it. Include enough theory to justify the project and clarify the goals. 

\vspace{1em}

\noindent
\textbf{Project Specification:} \\
This specification outlines the project's goals and deliverables. It should be clear enough for another person to carry out the work based solely on this document, without delving into required theory.

\vspace{1em}

\noindent
\textbf{Project Schedule:} \\
Identify the main tasks and work packages associated with the project. Outline a clear plan to reach the project’s completion, indicating interdependencies and critical paths. Include milestones in a Gantt chart.

\vspace{1em}

% New page
\newpage

\noindent
\textbf{Risk Register:} \\
Identify key risks that could delay project completion. Evaluate their likelihood and impact, and propose mitigation strategies. Consider if certain risks expire after specific project phases.

\begin{center}
    \begin{table}[ht]
        \centering
        \resizebox{\textwidth}{!}{%
            \begin{tabular}{|c|c|c|c|c|} 
                \hline
                \textbf{No.} & \textbf{Description of Risk} & \textbf{Risk Evaluation (L/M/H)} & \textbf{Chance of Risk Occurring (L/M/H)} & \textbf{Mitigation of Risk} \\ 
                \hline
                1 & Loss of data & M & L & Multiple backups \\ 
                \hline
            \end{tabular}
        }
    \end{table} 
\end{center}

\vspace{1em}

\end{document}
